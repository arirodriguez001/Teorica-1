% Documento base de LaTeX
\documentclass[12pt,a4paper]{article}
\usepackage[utf8]{inputenc}
\usepackage[T1]{fontenc}
\usepackage{lmodern}
\usepackage{amsmath,amssymb}
\usepackage{geometry}
\geometry{margin=2.5cm}
\usepackage{hyperref}
\usepackage{setspace}
\singlespacing

\setlength{\abovedisplayskip}{6pt plus 2pt minus 2pt}
\setlength{\belowdisplayskip}{6pt plus 2pt minus 2pt}
\setlength{\jot}{4pt}

\setlength{\parskip}{0pt}
\setlength{\parindent}{1em}


\title{Solución General del Potencial - Separacion de Variables}
\author{ }
\date{\today}
\begin{document}

\maketitle

\section{Coordenadas Cartesianas}

\subsection{Base en $x,y$}

% Acotado
Espacio acotado 

\begin{equation}
\Phi(x,y,z)
=\sum_{n=1}^{\infty}\sum_{m=1}^{\infty}
\sin\Bigl(\frac{n\pi x}{a}\Bigr)
\sin\Bigl(\frac{m\pi y}{b}\Bigr)
\Bigl[
A_{nm}\,e^{+\kappa_{nm}z}
+ B_{nm}\,e^{-\kappa_{nm}z}
\Bigr],
\quad
\kappa_{nm}
=\sqrt{\Bigl(\frac{n\pi}{a}\Bigr)^{2}+\Bigl(\frac{m\pi}{b}\Bigr)^{2}}.
\end{equation}

% No acotado

Espacio no acotado

\begin{equation}
\Phi(x,y,z)
=\iint_{-\infty}^{\infty}
 e^{i(k_x x + k_y y)}
\Bigl[
A(k_x,k_y)\,e^{+\kappa z}
+ B(k_x,k_y)\,e^{-\kappa z}
\Bigr]
\frac{dk_x\,dk_y}{(2\pi)^2},
\quad
\kappa=\sqrt{k_x^2+k_y^2}.
\end{equation}

\subsection{Base en $x,z$}

% Acotado

Espacio acotado

\begin{equation}
\Phi(x,y,z)
=\sum_{n=1}^{\infty}\sum_{p=1}^{\infty}
\sin\Bigl(\frac{n\pi x}{a}\Bigr)
\sin\Bigl(\frac{p\pi z}{c}\Bigr)
\Bigl[
C_{np}\,e^{+\kappa'_{np}y}
+ D_{np}\,e^{-\kappa'_{np}y}
\Bigr],
\quad
\kappa'_{np}
=\sqrt{\Bigl(\frac{n\pi}{a}\Bigr)^{2}+\Bigl(\frac{p\pi}{c}\Bigr)^{2}}.
\end{equation}

% No acotado

Espacio no acotado

\begin{equation}
\Phi(x,y,z)
=\iint_{-\infty}^{\infty}
 e^{i(k_x x + k_z z)}
\Bigl[
C(k_x,k_z)\,e^{+\kappa' y}
+ D(k_x,k_z)\,e^{-\kappa' y}
\Bigr]
\frac{dk_x\,dk_z}{(2\pi)^2},
\quad
\kappa'=\sqrt{k_x^2+k_z^2}.
\end{equation}

\subsection{Base en $y,z$}

% Acotado

Espacio acotado

\begin{equation}
\Phi(x,y,z)
=\sum_{m=1}^{\infty}\sum_{p=1}^{\infty}
\sin\Bigl(\frac{m\pi y}{b}\Bigr)
\sin\Bigl(\frac{p\pi z}{c}\Bigr)
\Bigl[
E_{mp}\,e^{+\kappa''_{mp}x}
+ F_{mp}\,e^{-\kappa''_{mp}x}
\Bigr],
\quad
\kappa''_{mp}
=\sqrt{\Bigl(\frac{m\pi}{b}\Bigr)^{2}+\Bigl(\frac{p\pi}{c}\Bigr)^{2}}.
\end{equation}

% No acotado

Espacio no acotado

\begin{equation}
\Phi(x,y,z)
=\iint_{-\infty}^{\infty}
 e^{i(k_y y + k_z z)}
\Bigl[
E(k_y,k_z)\,e^{+\kappa'' x}
+ F(k_y,k_z)\,e^{-\kappa'' x}
\Bigr]
\frac{dk_y\,dk_z}{(2\pi)^2},
\quad
\kappa''=\sqrt{k_y^2+k_z^2}.
\end{equation}

\section{Coordenadas Cilíndricas}

En coordenadas cilíndricas \((\rho,\varphi,z)\) la separación da siempre un modo angular \(e^{i m\varphi}\) (\(m\in\mathbb Z\)) y una parte radial y otra en \(z\). 
\\En $\varphi$ siempre tenemos base

\subsection{Base en \(\varphi,\rho\)}

En esta base se toma
\(\;Q_\nu(\varphi)=e^{i\nu\varphi},\;\nu\in\mathbb Z\).

\paragraph{Espacio acotado: \(\rho\in[0,a]\), \(\varphi\in[0,2\pi]\)}  
Se imponen \(R(\rho=a)=0\), de modo que los ceros discretos vienen de  
\(\alpha_{\nu n}\) = \(n\)-ésimo cero de \(J_\nu\). Definimos  
\(\kappa_{\nu n}=\alpha_{\nu n}/a\).  
\[
\boxed{
\Phi(\rho,\varphi,z)
=\sum_{\nu=-\infty}^{\infty}\sum_{n=1}^{\infty}
\bigl[
E_{\nu n}\,e^{+\kappa_{\nu n}\,z}
+F_{\nu n}\,e^{-\kappa_{\nu n}\,z}
\bigr]\,
J_{\nu}\!\bigl(\alpha_{\nu n}\,\tfrac{\rho}{a}\bigr)\,
e^{i\nu\varphi}.
}
\]

\paragraph{Espacio no acotado: \(\rho\in[0,\infty)\), \(\varphi\in[0,2\pi]\)}  
El espectro radial es ahora continuo \(k\ge0\); definimos \(\kappa=k\).  
\[
\boxed{
\Phi(\rho,\varphi,z)
=\sum_{\nu=-\infty}^{\infty}
\int_{0}^{\infty}
\bigl[
E_{\nu}(k)\,e^{+k\,z}
+F_{\nu}(k)\,e^{-k\,z}
\bigr]\,
J_{\nu}(k\,\rho)\,
e^{i\nu\varphi}\,
\frac{k\,dk}{2\pi}.
}
\]
\subsection{Base en \(\varphi,z\)}

Aquí se separa \(\varphi\) y \(z\), quedando \(\rho\) como variable restante.

\paragraph{Espacio acotado: \(\varphi\in[0,2\pi]\), \(z\in[0,c]\)}  
Modos angulares \(e^{i\nu\varphi}\) y longitudinales \(\sin(n\pi z/c)\).  
Para finitud en \(\rho=0\) y ceros en \(\rho=a\), usamos ceros de \(I_\nu\) o \(K_\nu\) según la BC.  
Definimos \(\beta_{\nu n}\) = \(n\)-ésimo cero de la función radial en \(\rho=a\).  
\[
\boxed{
\Phi(\rho,\varphi,z)
=\sum_{\nu=-\infty}^{\infty}\sum_{n=1}^{\infty}
D_{\nu n}\,
I_{\nu}\!\bigl(\beta_{\nu n}\,\rho\bigr)\,
e^{i\nu\varphi}\,
\sin\!\Bigl(\tfrac{n\pi z}{c}\Bigr).
}
\]

\paragraph{Espacio no acotado: \(\varphi\in[0,2\pi]\), \(z\in(-\infty,\infty)\)}  
Para decaimiento radial usamos los \(K_\nu\).  
\[
\boxed{
\Phi(\rho,\varphi,z)
=\sum_{\nu=-\infty}^{\infty}
\int_{-\infty}^{\infty}
G_{\nu}(k_z)\,
K_{\nu}\bigl(\kappa\,\rho\bigr)\,
e^{i\nu\varphi}\,
e^{i k_z z}\,
\frac{dk_z}{2\pi},
\quad
\kappa=\sqrt{k_z^2}.
}
\]

\section{Coordenadas Esféricas}

En coordenadas esféricas \((r,\theta,\varphi)\) se separa en
\(\Phi\sim Q_m(\varphi)\,\Theta_{\ell m}(\theta)\,R(r)\),
con \(Q_m(\varphi)=e^{\,i m\varphi}\), \(\;m\in\mathbb Z\), y
\(\Theta_{\ell m}(\theta)=P_\ell^m(\cos\theta)\), \(\ell\ge |m|\).

\subsection{Base en \(\varphi,\theta\)}

\paragraph{Espacio acotado: \(\varphi\in[0,2\pi]\), \(\theta\in[0,\pi]\)}  
Modos angulares discretos:
\[
Q_m(\varphi)=e^{i m\varphi},\quad
\Theta_{\ell m}(\theta)=P_\ell^m(\cos\theta),
\quad m=-\ell,\dots,\ell,\;\ell=0,1,2,\dots
\]
La parte radial satisface
\[
r^2R''+2rR'-\ell(\ell+1)R-\lambda\,r^2R=0.
\]
\[
\boxed{
\Phi(r,\theta,\varphi)
=\sum_{\ell=0}^{\infty}\sum_{m=-\ell}^{\ell}
\bigl[A_{\ell m}\,r^{\ell}+B_{\ell m}\,r^{-(\ell+1)}\bigr]\,
P_\ell^m(\cos\theta)\,e^{i m\varphi}.
}
\]

\paragraph{Espacio no acotado: \(\varphi\in[0,2\pi]\), \(\theta\in[0,\pi]\)}  
Para el caso Helmholtz \(\lambda=-k^2\):
\[
\boxed{
\Phi(r,\theta,\varphi)
=\sum_{\ell=0}^{\infty}\sum_{m=-\ell}^{\ell}
\int_{0}^{\infty}
\bigl[A_{\ell m}(k)\,j_\ell(k r)+B_{\ell m}(k)\,n_\ell(k r)\bigr]\,
P_\ell^m(\cos\theta)\,e^{i m\varphi}\,
\frac{2k^2\,dk}{\pi}.
}
\]
Para Helmholtz modificado \(\lambda=+\kappa^2\), reemplaza \(j_\ell,n_\ell\) por \(i_\ell,k_\ell\).

\subsection{Base en \(\varphi,r\)}

\paragraph{Espacio acotado: \(\varphi\in[0,2\pi]\), \(r\in[0,R]\)}  
Modos azimutales y discretos en \(r\):
\[
Q_m(\varphi)=e^{i m\varphi},\quad
R_{m n}(r)=j_m\!\bigl(\alpha_{m n}\,\tfrac{r}{R}\bigr),
\]
donde \(\alpha_{m n}\) es el \(n\)-ésimo cero de \(j_m\).  
La parte \( \Theta(\theta)\) satisface
\(\;(1/\sin\theta)(\sin\theta\,\Theta')'+[\ell(\ell+1)]\Theta=0\), así:
\[
\boxed{
\Phi(r,\theta,\varphi)
=\sum_{m=-\infty}^{\infty}\sum_{n=1}^{\infty}
C_{m n}\,
j_m\!\bigl(\alpha_{m n}\tfrac{r}{R}\bigr)\,
e^{i m\varphi}\,
P_m(\cos\theta).
}
\]

\paragraph{Espacio no acotado: \(\varphi\in[0,2\pi]\), \(r\in[0,\infty)\)}  
Aquí \(r\) continuo y \(\Theta=P_m\):
\[
\boxed{
\Phi(r,\theta,\varphi)
=\sum_{m=-\infty}^{\infty}
\int_{0}^{\infty}
D_{m}(k)\,
h_m^{(1)}(k r)\,
e^{i m\varphi}\,
P_m(\cos\theta)\,
\frac{2k^2\,dk}{\pi},
}
\]
con \(h_m^{(1)}\) ondas salientes (o usa \(h_m^{(2)}\) para entrantes).




\end{document}
